\usepackage[utf8]{inputenc}
\usepackage[T1]{fontenc}
\usepackage{graphicx}
\usepackage{subcaption}
\usepackage[backend=biber, style = numeric]{biblatex}
\addbibresource{Resurgence.bib}
\usepackage{amsmath,amsfonts,amsthm} % Math packages for equations
\usepackage{mathtools} % xarrow
\usepackage{todonotes}
\usepackage{tikz}
\usetikzlibrary{matrix}
\usepackage{epigraph}
\usepackage{physics}
\usepackage{cprotect} % verb in footnotes
\usepackage{empheq}
\usepackage{booktabs} % \toprule etc
\usepackage{cancel} % \cancel
\usepackage{url}
\usepackage{hyperref}
\usepackage{braket}
\usepackage{wrapfig}

%\usepackage{microtype} % Better typography

\setlength {\marginparwidth }{2cm} 



%----------------------------------------------------------------------------------------
%	Theorems and Definitions
%   https://tex.stackexchange.com/questions/45817/theorem-definition-lemma-problem-numbering
%----------------------------------------------------------------------------------------

\newtheoremstyle{break}%                % Name
  {}%                                     % Space above
  {}%                                     % Space below
  {}%                             % Body font
  {}%                                     % Indent amount
  {\itshape\bfseries}%                            % Theorem head font
  {}%                                    % Punctuation after theorem head
  {\newline}%                                    % Space after theorem head, ' ', or \newline
  {}%                                     % Theorem head spec (can be left empty, meaning `normal')

\newtheoremstyle{nte}%                % Name
  {}%                                     % Space above
  {}%                                     % Space below
  {}%                             % Body font
  {}%                                     % Indent amount
  {\itshape}%                            % Theorem head font
  {:}%                                    % Punctuation after theorem head
  { }%                                    % Space after theorem head, ' ', or \newline
  {}%                                     % Theorem head spec (can be left empty, meaning `normal')


\newtheoremstyle{trm}%                % Name
  {}%                                     % Space above
  {}%                                     % Space below
  {}%                             % Body font
  {}%                                     % Indent amount
  {\itshape}%                            % Theorem head font
  {:}%                                    % Punctuation after theorem head
  {\newline}%                                    % Space after theorem head, ' ', or \newline
  {}%                                     % Theorem head spec (can be left empty, meaning `normal')

\theoremstyle{break}

\renewenvironment{proof}{{\bf {Proof:} }}{\hfill $\Box$ \\} 
\newtheorem{Theorem}{Theorem}[section] % reset theorem numbering for each chapter
\newtheorem{definition}[Theorem]{Definition} % definition numbers are dependent on theorem numbers
%\newtheorem{exmple}[Theorem]{Example} % same for example numbers
\newtheorem{crllry}[Theorem]{Corollary} % same for example numbers
%\newtheorem{proof}[theorem]{Proof} % 


\theoremstyle{nte}
\newtheorem*{nte}{Note}
\newtheorem*{terminol}{Terminology}

%\theoremstyle{trm}
%\newtheorem*{terminology}{Terminology}


%----------------------------------------------------------------------------------------
%	EXAMPLE ENVIRONMENT
%   https://tex.stackexchange.com/questions/21227/example-environment/21241
%----------------------------------------------------------------------------------------
\usepackage[most]{tcolorbox}
\newcounter{examples}

\def\exampletext{Example} % If English

\NewDocumentEnvironment{example}{ O{} }
{
\colorlet{colexam}{red!55!black} % Global example color
\newtcolorbox[use counter=examples]{examplebox}{%
    % Example Frame Start
    empty,% Empty previously set parameters
    title={\exampletext: #1},% use \thetcbcounter to access the testexample counter text
    % Attaching a box requires an overlay
    attach boxed title to top left,
       % Ensures proper line breaking in longer titles
       minipage boxed title,
    % (boxed title style requires an overlay)
    boxed title style={empty,size=minimal,toprule=0pt,top=4pt,left=3mm,overlay={}},
    coltitle=colexam,fonttitle=\bfseries,
    before=\par\medskip\noindent,parbox=false,boxsep=0pt,left=3mm,right=0mm,top=2pt,breakable,pad at break=0mm,
       before upper=\csname @totalleftmargin\endcsname0pt, % Use instead of parbox=true. This ensures parskip is inherited by box.
    % Handles box when it exists on one page only
    overlay unbroken={\draw[colexam,line width=.5pt] ([xshift=-0pt]title.north west) -- ([xshift=-0pt]frame.south west); },
    % Handles multipage box: first page
    overlay first={\draw[colexam,line width=.5pt] ([xshift=-0pt]title.north west) -- ([xshift=-0pt]frame.south west); },
    % Handles multipage box: middle page
    overlay middle={\draw[colexam,line width=.5pt] ([xshift=-0pt]frame.north west) -- ([xshift=-0pt]frame.south west); },
    % Handles multipage box: last page
    overlay last={\draw[colexam,line width=.5pt] ([xshift=-0pt]frame.north west) -- ([xshift=-0pt]frame.south west); },%
    }
\begin{examplebox}}
{\end{examplebox}\endlist}
