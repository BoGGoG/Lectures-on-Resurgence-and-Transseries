\documentclass{exam}
\usepackage[utf8]{inputenc}
\usepackage{amsmath,amsfonts,amsthm} % Math packages for equations
\usepackage{tikz}

\DeclareMathOperator{\diff}{d}

\title{Lectures on Resurgence and Trans-Series}
\author{Marco Knipfer, University of Alabama}
\date{Week 3, \today}

\begin{document}
\maketitle


\begin{questions}
    \question (\textbf{$\lambda \phi^4$ - integral})\\
    Let's look at the integral
    \begin{equation}
        Z(\lambda) = \int_{-\infty}^\infty \diff\! x\, e^{-x^2 - \lambda x^4}\,.
        \label{eq:lambdaphi4}
    \end{equation}
    \begin{parts}
        \part (\textbf{Asymptotic expansion})
        \label{part:asymptExp}
        Expand the $\exp(-\lambda x^4)$ part in a power series around $\lambda = 0$,
        add a spurious factor $a$ to the $x^2$ part (in the end $a\to 1$)
        and change $x^4 \to \partial_a^2$.
        Finally, do the Gaussian integral to get
        \begin{equation}
            Z(\lambda) = \sum_{k = 0}^{\infty} \frac{(-\lambda)^k}{k!} 
            \partial_a^{2n} \sqrt{\frac{\pi}{a}} \bigg|_{a\to 1}\,.
            \label{eq:derivatives}
        \end{equation}

        \part (\textbf{Performing the derivative}) Show that the above eq.(\ref{eq:derivatives}) can be written
        in the form of an oscillating sum,
        \begin{equation}
            Z(\lambda) = \sum_{k = 0}^{\infty} \sqrt{\pi}
            \frac{(4k)!}{2^{4k} (2k)! k!} (-\lambda)^k =
            \sum_{k=0}^\infty c_k (-\lambda)^k\,.
            \label{eq:lambdaphi4expansion}
        \end{equation}

        \part (\textbf{Large order Behavior}) Use Stirling's approximation
        \begin{equation}
            k! \approx \frac{1}{\sqrt{2} \pi} 4^k k!, \quad (z \gg 1)\,,
        \end{equation}
        to show that
        \begin{equation}
            \sqrt{\pi} \frac{(4k)!}{2^{4k} (2k)! k!} \approx \frac{1}{\sqrt{2\pi}} 4^k k!\,.
        \end{equation}
        Why does this mean that the expansion eq.~(\ref{eq:lambdaphi4expansion}) diverges?

        \part (\textbf{Optional: Alternative derivation of the series})
        Don't use the "derivative trick" from part~\ref{part:asymptExp}, but just expand the
        exponential $e^{-\lambda x^4}$, exchange sum and integral and solve the integral by
        substitution and use of the definition of the $\Gamma$-function/factorial or just type it into Mathematica.

        %\part (\textbf{Gevrey order}) What is the Gevrey order of the resulting series?
        \part (\textbf{Comparison with exact result and optimal truncation}) The integral can actually be performed analytically,
        \begin{equation}
            Z(\lambda) = \frac{e^{\frac{1}{8\lambda}}}{2\sqrt{\lambda}} K_{1 / 4}\left(\frac{1}{8\lambda}\right)\,,
        \end{equation}
        with the modified Bessel function of the second kind $K_n(x)$.
        Take eq.~(\ref{eq:lambdaphi4expansion}) and $\lambda = \frac{1}{50}$.
        Truncate the sum~(\ref{eq:lambdaphi4expansion}) at $N$ and let's call this truncated sum
        $Z(\lambda; N)$. Plot $Z(\frac{1}{50}, N)$ against $N$ for $N = 1\ldots 40$ and also
        show the exact value $Z(\frac{1}{50})$.
        For $c_n \sim A^{-n} n!$ the optimal truncation is around\footnote{take some closest integer, I don't think it matters if you round up or down, since it's only an estimate anyways.}
        \begin{equation}
            N_* = \left|\frac{A}{\lambda} \right|\,.
        \end{equation}
        Do you find this here?
        \part (\textbf{Perturbative ambiguity}) The closest we can get to the exact result by optimal truncation scales like
        \begin{equation}
            \epsilon(\lambda) \sim e^{-|A / \lambda|}\,.
        \end{equation}
        Why can't we just add a term like this in the power series~(\ref{eq:lambdaphi4expansion})
        to get rid of the errors?
        \part (\textbf{Scaling of $\lambda$}) What happens to the instanton action $A$ if you
        scale $\lambda$ in eq.~(\ref{eq:lambdaphi4}) to $\alpha \lambda$, $\alpha >0$?
        What happens to the perturbative ambiguity?
    \end{parts}
    
    \question (\textbf{Analytic Continuation})
    In this question we will look at the function
    \begin{equation}
        f(z) = \sum_{n = 0}^\infty z^n,\quad z \in \mathbb{C}\,.
        \label{eq:f}
    \end{equation}
    \begin{parts}
        \part (\textbf{Radius of Convergence})
        Using the ratio test find the radius of convergence for $f(z)$. 
        \part (\textbf{Expansion around $z = -1 / 2$}) The function $f(z)$ can actually be represented as
        \begin{equation}
            f_\text{master}(z) = \frac{1}{1 - z}\,,
            \label{eq:fmaster}
        \end{equation}
        which has a singularity at $z=1$.
        This representation is called \emph{master representation}\footnote{at least I think it is. If it is not, it is now.}, because it works everywhere
        except at $z=1$. It is the analytic continuation of $f(z)$ to all of $\mathbb{C}\backslash \{1\}$.
        Eq.~(\ref{eq:f}) is the expansion of $f_\text{master}(z)$ around $z=0$.
        Expand $f_\text{master}$ around $z = -\frac{1}{2}$ and calculate the radius of convergence
        using the ratio test.
        \part (\textbf{Without Master Representation}) Without using $f_\text{master}$, calculate the Taylor series of $f(z)$ around
        $z = -\frac{1}{2}$ by taking the derivatives of the series.
        You should get the same as above.

        Tipp: Plug the series you get for $f^{(n)}(- \frac{1}{2})$ into Mathematica.
        \part (\textbf{Calculating $2 + 4 + 8 + \ldots$})
        In principle, how would you analytically continue $f(z)$ to $z = 2$ if you did not know the
        master representation?
        Calculate $f_\text{master}(2)$. 
        Does this make sense in terms of the original function~(\ref{eq:f})?

        Tipp: There is only the singularity at $z = 1$ and every expansion around any point
        $z_* \neq 1$ will have radius of convergence until $z = 1$, see the figure below.

        \begin{center}
        \begin{tikzpicture}
            \tikzset{
                cross/.pic = {
                    \draw[rotate = 45] (-#1,0) -- (#1,0);
                    \draw[rotate = 45] (0,-#1) -- (0, #1);
                }
            }
            % Configurable parameters
            \def\gap{0}
            \def\bigradius{3}
            \def\littleradius{0.5}

            % Axes
            \draw [help lines,->] (-1.25*\bigradius, 0) -- (1.25*\bigradius,0);
            \draw [help lines,->] (0, -1.25*\bigradius) -- (0, 1.25*\bigradius);
            % Red path
            \draw (0,0) circle (1); 
            \draw (-0.5,0) circle (1.5); 
            \draw (0.5,-1) circle (1.1); 
            \draw (1.2,-0.5) circle (0.5); 
            \draw (1,0) pic {cross=3pt}; 
        \end{tikzpicture}   
        \end{center}
        
    \end{parts}

    

    %\question (\textbf{$\lambda \phi^4$ - integral})\\
    %In this exercise we will examine the integral
    %\begin{align}
        %I(\hbar) &= \frac{1}{2\pi} \int_\Gamma \diff z e^{-\frac{1}{\hbar} V(z)}\,,\\
        %V(z) &= \frac{1}{2} z^2 - \frac{\lambda}{4!} z^4\,,
    %\end{align}
    %where $\Gamma$ is some contour in the complex plane.

    %\begin{parts}
        %\part (\textbf{Expansion in QFT-manner})
        %\begin{itemize}
            %\item Make a change of variables $z_\text{new} := \frac{z}{\sqrt{\hbar}}$ and
                %$x := \lambda \hbar$ to get
                %\begin{equation}
                    %Z(x) = \frac{\sqrt{\hbar}}{2\pi}\int_\Gamma \diff z e^{- \frac{1}{2} z^2 + \frac{x}{4!}z^4}\,.
                %\end{equation}
            %\item 
            %\item Calculate the saddle points $\rho(z_*) = 0$ of the new exponent 
                %$\rho(z) = - \frac{1}{2} z^2 + \frac{x}{24}z^4$.
                %First we will make an expansion around the \emph{perturbative} saddle
                %point $z_0 = 0$.
        %\end{itemize}

    %\end{parts}

    
\end{questions}

\end{document}
