\documentclass{exam}
\usepackage[utf8]{inputenc}
\usepackage{amsmath,amsfonts,amsthm} % Math packages for equations
\usepackage{tikz}
\usepackage{url}

\DeclareMathOperator{\diff}{d}
\newcommand{\calB}{\mathcal{B}}
\newcommand{\calL}{\mathcal{L}}

\title{Lectures on Resurgence and Trans-Series}
\author{Marco Knipfer, University of Alabama}
\date{Week 4, Version: \today}

\begin{document}
\maketitle

\begin{questions}

    \question (\textbf{Borel Summation of the $\lambda \phi^4$ Integral})\\
    The integral
    \begin{equation}
        Z(\lambda) = \int_{-\infty}^\infty \diff\!x\, e^{-x^2 - \lambda x^4}
    \end{equation}
    has the analytic solution
    \begin{equation}
        Z(\lambda)_\text{analytic} =
        \frac{e^{\frac{1}{8 \lambda }} K_{\frac{1}{4}}\left(\frac{1}{8 \lambda }\right)}{2 \sqrt{\lambda }}
    \end{equation}
    and, as we have checked on the last exercise sheet, the asymptotic series
    \begin{equation}
        Z(\lambda)_\text{asympt.} = \sqrt{\pi} \sum_{k=0}^\infty \frac{(4k)! (-1)^k}{2^{4k}(2k)!\, k!}\lambda^k\,.
        \label{eq:Zasympt}
    \end{equation}
    \begin{parts}
        \part ( \textsc{Mathematica}: \textbf{Asymptotic Expansion of the Analytic Solution})\\
    Hankel came up with an asymptotic expansion for the \emph{modified Hankel function}\footnote{also called \emph{Basset function, modified Bessel function of the third kind and Macdonald function}}
        $K_\alpha(z)$,
        \begin{equation}
            K_\alpha(z) \sim \sqrt{\frac{\pi}{2z}} e^{-z} \left(1 + \frac{4
                    \alpha^2 - 1}{8z} + \frac{\left(4 \alpha^2 - 1\right)
                    \left(4 \alpha^2 - 9\right)}{2! (8z)^2} + \frac{\left(4
                        \alpha^2 - 1\right) \left(4 \alpha^2 - 9\right) \left(4
            \alpha^2 - 25\right)}{3! (8z)^3} + \cdots \right) \,,
            \label{eq:besselAsympt}
        \end{equation}
        for $\left|\arg z\right|<\frac{3\pi}{2}$.\footnote{\url{https://en.wikipedia.org/wiki/Bessel_function#Asymptotic_forms}}
        Check that the asymptotic expansion of $Z(\lambda)_\text{analytic}$ is exactly
        $Z(\lambda)_\text{asympt.}$. 
        The easiest way is to let Mathematica expand $Z(\lambda)_\text{analytic}$ with
        \verb!Series[ ]! and compare the first few terms.
        Note that Mathematica knows about the asymptotic expansion~(\ref{eq:besselAsympt}).
        \part (\textbf{Borel Transform})\\
        The \emph{Borel transform} $\calB$ is defined as
        \begin{align}
            B &: z^{-1}\mathbb{C}[[z^{-1}]] \to \mathbb{C}[[\zeta]]\,,\\
            B &: \tilde{\phi}(z) = \sum_{n=0}^{\infty} c_n z^{-n-1}
            \mapsto \hat{\phi}(\zeta) = \sum_{n=0}^{\infty} c_n \frac{\zeta^n}{n!}\,.
            \label{eq:BorelDef}
        \end{align}
        Note that we are working with $\lambda = \frac{1}{z}$ here.
        The Borel transform is not defined for the constant ($k=0$) term in~(\ref{eq:Zasympt}).
        Take out the constant term $\sqrt{\pi}$ and write
        \begin{equation}
            Z(\lambda)_\text{asympt} = \sqrt{\pi} + \tilde{Z}(\lambda)_\text{asympt}\,.
            \label{eq:Ztild}
        \end{equation}
        Perform the borel transform $\calB[\tilde{Z}_\text{asympt}]$ by simply substituting
        \begin{equation}
            \lambda^k \to \frac{\zeta^{k-1}}{(k-1)!}\,.
        \end{equation}
        By the ratio test\footnote{\url{https://en.wikipedia.org/wiki/Ratio_test}},
        figure out for what $\lambda$ $\calB[\tilde{Z}_\text{asympt.}](\zeta)$ converges.
        \part (\textsc{Mathematica}: \textbf{Analytic Continuation})\\
        Type $\calB[\tilde{Z}_\text{asympt.}]$ as an infinite sum into Mathematica,
        it should be able to write it as a Hypergeometric function.
        \begin{equation}
            \calB[\tilde{Z}_\text{asympt}](\zeta) = 
            -\frac{3\sqrt{\pi}}{4} \, _2F_1\left(\frac{5}{4},\frac{7}{4},2\bigg|-4 \zeta \right)\,.
            \label{eq:borelAnalyticCont}
        \end{equation}
        The Hypergeometric function $_2F_1(a,b,c|z)$ has a singularity at $z=1$ and a branch cut from
        $z=1$ to infinity.
        Note that~(\ref{eq:borelAnalyticCont}) is the full analytic continuation of our Borel transform,
        since it works on all of $\mathbb{C}\backslash\left\{-\frac{1}{4}\right\}$.
        Is the singularity at $z=1$ a problem for the Laplace transform?

        \part (\textsc{Mathematica}: \textbf{Inverse Borel Transform / Laplace Transform})\\
        Perform the Laplace transform
        \begin{equation}
            \calL^\theta[\hat{\phi}](z) = \int_0^{e^{i\theta}\infty}\diff\! \zeta \,
            e^{-z\zeta} \hat{\phi}(\zeta)\,,
        \end{equation}
        with $\theta = 0$ in Mathematica and compare the result with $Z(\lambda)_\text{analytic}$.
        Don't forget to add the constant term again that we subtracted in~(\ref{eq:Ztild}).
        
        

        

    \end{parts}
    
    
    

    

    %\question (\textbf{$\lambda \phi^4$ - integral})\\
    %In this exercise we will examine the integral
    %\begin{align}
        %I(\hbar) &= \frac{1}{2\pi} \int_\Gamma \diff z e^{-\frac{1}{\hbar} V(z)}\,,\\
        %V(z) &= \frac{1}{2} z^2 - \frac{\lambda}{4!} z^4\,,
    %\end{align}
    %where $\Gamma$ is some contour in the complex plane.

    %\begin{parts}
        %\part (\textbf{Expansion in QFT-manner})
        %\begin{itemize}
            %\item Make a change of variables $z_\text{new} := \frac{z}{\sqrt{\hbar}}$ and
                %$x := \lambda \hbar$ to get
                %\begin{equation}
                    %Z(x) = \frac{\sqrt{\hbar}}{2\pi}\int_\Gamma \diff z e^{- \frac{1}{2} z^2 + \frac{x}{4!}z^4}\,.
                %\end{equation}
            %\item 
            %\item Calculate the saddle points $\rho(z_*) = 0$ of the new exponent 
                %$\rho(z) = - \frac{1}{2} z^2 + \frac{x}{24}z^4$.
                %First we will make an expansion around the \emph{perturbative} saddle
                %point $z_0 = 0$.
        %\end{itemize}

    %\end{parts}

    
\end{questions}

\end{document}
