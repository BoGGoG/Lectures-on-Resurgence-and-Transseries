\documentclass{exam}
\usepackage[utf8]{inputenc}
\usepackage{amsmath,amsfonts,amsthm} % Math packages for equations
\usepackage{tikz}
\usepackage{url}

\DeclareMathOperator{\diff}{d}
\DeclareMathOperator{\sign}{sgn}
\DeclareMathOperator*{\Res}{Res}
\DeclareMathOperator*{\reg}{reg}
\DeclareMathOperator*{\Ai}{Ai}
\DeclareMathOperator*{\Bi}{Bi}
\newcommand{\calB}{\mathcal{B}}
\newcommand{\calL}{\mathcal{L}}

\title{Lectures on Resurgence and Trans-Series}
\author{Marco Knipfer, University of Alabama}
\date{Week 8, Version: \today}

\begin{document}
\maketitle

\begin{questions}
    \setcounter{question}{7}
    \question (\textbf{The Airy Equation: WKB})\\
    The Airy equation
    \begin{equation}
        y''(x) = x y(x)\,,
        \label{eq:Airy}
    \end{equation}
    has a transseries solution which we are going to develop here.
    \begin{parts}
        \part (\textbf{Neglecting the wrong term})\\
        We are interested in the asymptotics as $x\to\infty$.
        Because the Airy equation has an irregular singular point at $x=\infty$, the ansatz is
        \begin{equation}
            y(x) = A e^{S(x)}\,.
            \label{eq:trafo}
        \end{equation}
        Then the Airy equation~(\ref{eq:Airy}) becomes
        \begin{equation}
            S'' + (S')^2 = x\,.
            \label{eq:AiryS}
        \end{equation}
        Try neglecting $(S')^2$, \emph{i.e.}\ assuming $(S')^2\ll x, S''$ as $x \to \infty$.
        Then we would write equation~(\ref{eq:AiryS}) as $S'' \sim x$ as $x\to\infty$.
        Calculate $S$ in this approximation and show that the assumption is not consistent.

        \part (\textbf{WKB dominant part})\\
        Neglect $S''$, \emph{i.e.}\ assume $S'' \ll x, (S')^2$ as $(x\to\infty)$. 
        Calculate $S(x)$ in this approximation and show that it is consistent with the assumption.

        As bonus you can also try neglecting $x$. 
        Generally only one version will be consistent.

        \part (\textbf{What is usually called WKB})\\
        You should now have $S\sim \pm \frac{2}{3} x^{\frac{3}{2}}$.
        One of the few things one generally cannot do with an asymptotic expression like this is to
        exponentiate it, which is exactly what we need, see eq.~(\ref{eq:trafo}).
        We first have to calculate more terms for $S$.
        This can be done by writing
        \begin{equation}
            S(x) = \pm \frac{2}{3} x^{\frac{3}{2}} + S_2(x)\,,
            \label{eq:S_2}
        \end{equation}
        with $S_2(x) \ll x^{\frac{3}{2}}$, $S_2'(x)\ll \sqrt{x} $, $S_2''(x) \ll \frac{1}{\sqrt{x}}$ as $x\to\infty$.
        Generally one can integrate asymptotic relations, but not differentiate (try $\sin(x^{25})$).
        Here however it works because $S$ comes from a differential equation.
        Notice there is an equals sign in (\ref{eq:S_2}), no asymptotic.

        Plug $S(x)$ from (\ref{eq:S_2}) into (\ref{eq:AiryS}), find the right terms to throw away and
        calculate $S_2(x)$ asymptotically.
        Show that your solution is consistent with the assumptions.

        \part (\textbf{Next Term (last one, I promise!)})\\
        You should now have $S \sim \pm \frac{2}{3} x^{\frac{3}{2}} - \frac{1}{4} \ln(x)$.
        Again replace the asymptotic sign with an equal sign and write
        \begin{equation}
            S(x) = \pm \frac{2}{3} x^{\frac{3}{2}} - \frac{1}{4} \ln(x) + S_3(x)\,,
        \end{equation}
        where $S_3(x)$ is much smaller than the other terms as $x\to\infty$.
        Do the same as above and calculate $S_3(x)$ asymptotically.
        Show that what you did is consistent.

        Since $S_3(x)$ goes to zero as $x\to\infty$ we can now exponentiate:
        \begin{equation}
            y \sim C_\pm \frac{e^{\pm\frac{2}{3}x^{\frac{3}{2}}}}{x^{1 / 4}} \left( 1 + S_3(x) + \ldots\right)\,.
        \end{equation}

        \part (\textbf{Full asymptotic series for the Airy function})\\
        \label{part:AiryFunction}
        The full asymptotic series\footnote{The full function is $\Ai(x)$, the leading asymptotic behavior is
        $\Ai(x)\sim \frac{1}{2\sqrt{\pi}} \frac{e^{-\frac{2}{3} x^{\frac{3}{2}}}}{x^{1 /4}}$ and the full
        asymptotic series is $Z_{Ai}$.} is
        \begin{align}
            Z_{\Ai}(x) &\sim \frac{1}{2\sqrt{\pi}} \frac{e^{- \frac{2}{3} x^{\frac{3}{2}}}}{x^{1 / 4}}
            \sum_{n=0}^\infty a_n x^{-\frac{3}{2}n}\,,\quad x \to \infty
            \intertext{with}
            a_n &= \frac{1}{2\pi} \left(- \frac{3}{4}\right)^n \frac{\Gamma(n+\frac{5}{6}) \Gamma(n + \frac{1}{6})}{n!}\,.
        \end{align}
        This is the Airy function\footnote{The Airy function is one solution of the Airy equation.}
        and the prefactor $\frac{1}{2 \sqrt{\pi} }$ can be seen as the definition of this particular solution.

        Show that the coefficients $a_n$ grow as
        \begin{equation}
            a_n \sim A^{-n} n!\,,
        \end{equation}
        and find the ``instanton action'' $A$.

        \part (\textbf{The Bairy function})\\
        There is another independent solution that we just call the ``Bairy function\footnote{
        I know what you're thinking. No, there is no $\mathrm{Hi}(x)$.}'' $\Bi(x)$
        and its formal power series solution around $x=\infty$ is
        \begin{equation}
            Z_{\Bi}(x) \sim \frac{1}{\sqrt{\pi}} \frac{e^{+ \frac{2}{3} x^{\frac{3}{2}}}}{x^{1 / 4}}
            \sum_{n=0}^\infty (-1)^n a_n x^{-\frac{3}{2}n}\,.
        \end{equation}
        Notice the different prefactor and that the series is not oscillating ($(-1)^n$ is also in the coefficients $a_n$).

        The general transseries solution to the Airy equation~(\ref{eq:Airy}) is now
        \begin{equation}
            C_1 Z_{\Ai} (x) + C_2 Z_{Bi}(x)\,.
        \end{equation}
        Show that the \emph{relative exponential weight} between the two solutions is
        \begin{equation}
            e^{A x^{\frac{3}{2}}}\,,
        \end{equation}
        where $A$ is the instanton action from (\ref{part:AiryFunction}).
        
        \part (\textbf{Stokes Lines of the Bairy function})\\
        Write $x=r e^{i \phi}$. 
        Stokes lines appear where the real part of the exponential is zero.
        Find the Stokes lines for $\Bi(x)$.
        The leading asymptotic behavior 
        \begin{equation}
            \Bi(x) \sim \frac{1}{\sqrt{\pi}} \frac{1}{x^{1 / 4}} e^{\frac{2}{3} x^{\frac{3}{2}}}\,,\quad
            x\to \infty,\ |\arg x| < \frac{\pi}{3}\,,
        \end{equation}
        is good only on the Stokes wedge $|\arg x| < \frac{\pi}{3}$.
        Thus in general one cannot just analytically continue the leading asymptotic behavior through Stokes lines.
        It is however possible to continue the full asymptotic series past Stokes lines!
        On the next exercise sheet we will find out that even though $\Ai$ is good until
        $\arg x = \pi$ even though it also has Stokes lines at $|\arg x| = \frac{\pi}{3}$.
        
    \end{parts}
    
\end{questions}


\end{document}
