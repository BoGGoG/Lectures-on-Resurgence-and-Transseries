\documentclass{exam}
\usepackage[utf8]{inputenc}
\usepackage{amsmath,amsfonts,amsthm} % Math packages for equations
\usepackage{tikz}
\usepackage{url}

\DeclareMathOperator{\diff}{d}
\DeclareMathOperator{\sign}{sgn}
\DeclareMathOperator*{\Res}{Res}
\newcommand{\calB}{\mathcal{B}}
\newcommand{\calL}{\mathcal{L}}

\title{Lectures on Resurgence and Trans-Series}
\author{Marco Knipfer, University of Alabama}
\date{Week 7, Version: \today}

\begin{document}
\maketitle

\begin{questions}
    \setcounter{question}{5}
    \question (\textsc{Warmup}: \textbf{Euler's Equation})\\
    The differential equation
    \begin{equation}
        \phi'(z) - \phi(z) = -\frac{1}{z}\,,
        \label{eq:euler}
    \end{equation}
    is not really called Euler's equation, but he studied it.
    Here we are working in the $z\to \infty$ limit.

    \begin{parts}
        \part (\textbf{Perturbative Solution})\\
        Show that the alternating (asymptotic) series
        \begin{equation}
            \tilde{\phi}(z) = \sum_{n=0}^{\infty} (-1)^n n!\, z^{-n-1}\,,
            \label{eq:perturbEuler}
        \end{equation}
        solves eq.~(\ref{eq:euler}).

        \part (\textbf{Borel Transformation and Analytic Continuation})\\
        Perform the Borel transform
        \begin{align}
            \tilde{\phi}(z) &= \sum_{n = 0}^\infty a_n z^{-n-1} \in z^{-1}\mathbb{C}[[z^{-1}]]\,,\\
            \hat{\phi}(z) &= \mathcal{B}[\tilde{\phi}][\zeta] = \sum_{n=0}^\infty \frac{a_n}{n!}\zeta^n\,.
            \label{eq:borel}
        \end{align}
        on the perturbative solution~(\ref{eq:perturbEuler}).
        Perform an analytic continuation of the Borel transform by sharply looking at it
        and realizing that it's a geometric series. 
        Where are the poles of the analytic continuation of the Borel transform?

        \part (\textsc{Optional}: \textbf{Laplace transform})\\
        Do the Laplace transform and find
        \begin{equation}
            \calL^0[\hat{\phi}](z) = e^{z} \Gamma(0;z)\,.
        \end{equation}
        The most general solution is $e^{z} \Gamma(0;z) + c e^{z}$,
        which you can find by plugging the differential equation into Mathematica's \verb!DSolve!.
        Why can't we find the $c e^{z}$ term?
        Keep in mind that we are approximating around $z=\infty$.
        
    \end{parts}

    \question (\textbf{Modification of Euler's Equation})\\
    Let's change a sign,
    \begin{equation}
        \phi'(z) + \phi(z) = +\frac{1}{z}\,.
        \label{eq:euler2}
    \end{equation}
    Then the asymptotic series changes to
    \begin{equation}
        \tilde{\phi}(z) = \sum_{n=0}^{\infty} n!\, z^{-n-1}\,,
        \label{eq:perturbEuler2}
    \end{equation}
    which is no longer alternating.

    \begin{parts}
        \part (\textbf{Borel Transform and Analytic Continuation})\\
        Perform the Borel transform as in eq.~(\ref{eq:borel}) and use your magic powers
        to find the analytic continuation $\hat{\phi}(\zeta) = \frac{1}{1 - \zeta}$.
        Where is the pole and will it matter for the Laplace transform?

        \part (\textbf{Lateral Laplace Transform})\\
        Perform a lateral Laplace transform
        \begin{equation}
            \calL^\theta[\hat{\phi}](z) = \int_0^{\infty e^{i\theta}} \diff\!\zeta\,e^{-z\zeta} \hat{\phi}(\zeta)\,,
        \end{equation}
        by first making a change of variables, $\zeta = e^{i\theta} \xi$, and then asking
        Mathematica.

        \part{ (\textbf{Ambiguity is Purely Imaginary})}\\
        By \emph{e.g.}\@ using \verb!ReImPlot! in Mathematica, show that for $z\in \mathbb{R}$
        the lateral Laplace transforms for $\frac{\pi}{2}>\theta>0$ and
        $0>\theta>-\frac{\pi}{2}$ have the same real part, but differ in the imaginary part.
        This is by the way a general feature of this Borel-Laplace stuff.

        Derive this ambiguity by the resiude theorem.
        The two directions enclose the singularity and the part at infinity vanishes,
        thus with the notation 
        \begin{align*}
            S_\theta \tilde{\phi} &\sim \calL^\theta[\calB[\tilde{\phi}]]\,,
            \intertext{one just has to calculate}
            (S_{0^+} - S_{0^-})\tilde{\phi}(z) &= - 2 \pi i \Res\limits_{\zeta \to 1}\left( \frac{e^{-z\zeta}}{1 - \zeta}\right)
        \end{align*}
        


        You can see that this ambiguity is non-analytic and can't be
        ``touched'' by the perturbative expansion.

        \part{ (\textbf{Median Summation})}\\
        Simply perform a \emph{median resummation} (for $\theta = 0$) defined as
        \begin{align*}
            S^\text{med}_\theta &\sim \frac{1}{2}\left(S_{\theta^+} + S_{\theta^-}\right)\,,
        \end{align*}
        in order to get rid of the imaginary part.
        Plot the Borel-Laplace (median resummed) solution, a truncation of the asymptotic series~(\ref{eq:perturbEuler2})
        and the analytic solution from \verb!DSolve!.
        
    \end{parts}

    
\end{questions}

\end{document}
